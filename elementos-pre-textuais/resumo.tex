Shader é um tipo de programa de computador utilizado para simular como a luz
interage com os objetos ou as superfícies. Ele torna possível criar aspectos visuais nas superfícies de objetos 3D. Por exigir bastante recursos computacionais, a performance desses programas é algo importante. Atualmente é exigida a renderização em um curto intervalo de tempo de gráficos cada vez mais realistas. Ao fazer uso de shaders custosos e não otimizados, podem ocorrer alguns problemas como surgimento de artefatos, incompatibilidade com hardwares e o superaquecimento da GPU. Nesse estudo foram utilizados três motores de jogo. O primeiro foi o Godot, um software para produção de jogos 2D e 3D criado no ano de 2007. O segundo foi Unity que é a escolha mais comum entre desenvolvedores de jogos profissionais e amadores por sua capacidade de prototipação rápida e pela ampla gama de plataformas-alvo. O terceiro foi Unreal Engine, da Epic Games, que é o mais robusto e também é muito utilizado tanto por desenvolvedores profissionais quanto iniciantes. A utilidade desse estudo consiste em descobrir qual game engine, utilizando critérios quantizados de performance, apresenta a melhor ferramenta para criação de Shaders. A pesquisa realizada para este trabalho é do tipo aplicada. É um estudo com características exploratórias e descritivas. A abordagem é do tipo quantitativa. Os dados necessários para a realização do \textit{benchmarking} foram obtidos pelas ferramentas de profiling, dada sua natureza de tempo real. Os dados foram analisados para os shaders utilizados no cenário padrão de testes em cada uma das game engines levando em consideração múltiplos níveis de otimização. Foi observado que a Unity mostrou-se como uma solução mais completa. A Unreal também ofereceu uma variedade de \textit{shaders} otimizados, porém suas ferramentas de análise de desempenho e opções de otimização deixaram a desejar. Os resultados obtidos com a Godot foram os piores dentre os três.

% Separe as palavras-chave por ponto
\palavraschave{Shaders. Unity. Unreal Engine. Godot. Otimização}