A shader is a computer program used to simulate how light interact with objects and surfaces. It makes it possible to create visual aspects on the surfaces of 3D objects. As it requires a lot of computational resources, the performance of these programs is something important. Currently, rendering increasingly realistic graphics in a short time span is required. When using costly and not optimized shaders, some problems can occur such as the appearance of artifacts, incompatibility with hardware and GPU overheating. In this study, three game engines were used. The first was Godot, a 2D and 3D game production software created in the year 2007. The second was Unity that is the most common choice among professional and amateur game developers for it's rapid prototyping capability and wide range of target platforms. The third was Unreal Engine, by Epic Games, which is the most robust and is also widely used by both professional developers and beginners. The usefulness of this study is to find out which game engine, using quantized performance criteria, presents the best tool for creating Shaders. The research carried out for this work is of the applied type. It is a study with exploratory and descriptive characteristics. The approach is of the quantitative type. The data obtained to perform the \textit{benchmarking} were obtained by profiling tools, given its real-time nature. Data were identified for the shaders used in the test scenario in each of the game engines taking into account optimization levels. It was observed that Unity proved to be a more complete solution. Unreal also offered a variety of optimized \textit{shaders}, however their performance analysis tools and optimization options falls short. The results obtained with Godot were the worst among the three.

% Separe as Keywords por ponto
\keywords{Shaders. Unity. Unreal Engine. Godot. Optimization}