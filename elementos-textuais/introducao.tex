\chapter{Introdução}
\label{cap:introducao}

Shader é um tipo de programa de computador utilizado para simular como a luz interage com os objetos ou as superfícies \cite{unityShaders}. Por meio de seu uso é possível criar aspectos visuais nas superfícies de objetos 3D, para que com o uso de texturas, seja possível obter uma aparência de metal ou de madeira, por exemplo.

Por demandar recursos computacionais da GPU em tempo real, a performance de execução desses programas é um assunto que requer atenção, ainda mais levando em consideração o avanço da tecnologia de computação gráfica, que exige a renderização em um curto intervalo de tempo de gráficos cada vez mais realistas. Quanto maior a frequência de realização de cálculos e processamentos durante esse processo, maior será o impacto na performance de um jogo. Ao fazer uso de shaders custosos e não otimizados, podem ocorrer alguns problemas como surgimento de artefatos, incompatibilidade com hardwares de gerações passadas e o superaquecimento da GPU devido a cargas muito altas de trabalho. 

Para realizar o desenvolvimento, a execução e o estudo de performance dos shaders, três dos mais populares motores de jogo foram escolhidos. O primeiro foi o Godot, um software para produção de jogos 2D e 3D criado no ano de 2007, quando seus desenvolvedores perceberam duas importantes mudanças no cenário de desenvolvimento de games: uma foi a melhoria de hardware disponível que permitiu que dispositivos portáteis ganhassem mais poder de processamento, a outra mudança foi na forma que as CPUs passaram a ser divididas em múltiplos núcleos, o que permitiu o advento do processamento paralelo \cite{godotEngine}.

O segundo motor de jogo, Unity, é a escolha mais comum entre desenvolvedores de jogos profissionais e amadores por sua capacidade de prototipação rápida e pela ampla gama de plataformas-alvo de compilação. Ela foi criada com os objetivos de fornecer uma engine de custo acessível com ferramentas profissionais e democratizar o acesso à indústria de desenvolvimento de games \cite{unityHistory}.

O terceiro motor de jogo escolhido foi a Unreal Engine, produzida pela Epic Games para desenvolvimento de jogos e aplicações, seja de grandes orçamentos e níveis de promoção, seja de editoras ou produtoras independentes e com baixo orçamento. É o mais robusto e também é muito utilizado tanto por desenvolvedores profissionais quanto iniciantes \cite{unrealEngine}.

\section{Justificativa}
\label{sec:justificativa}

O processo de criação de shaders pode vir a apresentar-se, dependendo do nível de complexidade exigido pela tarefa, como uma atividade custosa e que exige elevados recursos computacionais. Sendo assim, o estudo das ferramentas de criação de shaders é importante para definir processos de otimização de performance para que empresas, indivíduos ou entusiastas possam economizar tempo e recursos ao utilizar essas ferramentas. 

Cabe ressaltar que a execução de programas de shaders muito custosos pode acarretar em problemas como queda da taxa de quadros por segundo, travamentos durante a execução do programa e na pior hipótese danos permanentes ao hardware que acabam por prejudicar o utilizador final e que de maneira geral acarretam em uma má experiência de usuário. 

No contexto específico dos jogos eletrônicos, o uso de shaders não otimizados pode fazer com que o jogo torne-se lento e apresente travamentos. Essas são características que tornam um jogo não atrativo e que geram sensações negativas no usuário. Elas fazem com que ele perca o interesse e se sinta frustrado, sendo levado à compartilhar feedback negativo, cujo acaba por prejudicar a imagem e as vendas do produto. Isso tem como consequência motivar outros possíveis usuários a não comprarem o jogo, principalmente aqueles que não possuem hardware compatível.

Nesse caso, a utilidade desse estudo consiste em descobrir qual game engine, utilizando critérios quantizados de performance, apresenta a melhor ferramenta para criação de Shaders. Além disso, shaders otimizados tornam-se favoráveis para serem aplicados para um público maior por ampliar a possibilidade de hardware compatível, ou seja, os jogos ou aplicações que fazem uso desse recurso conseguem ter um alcance maior e mais vendas.

\section{Objetivos}
\label{sec:objetivos}

\subsection{Objetivo Geral}
\label{sec:objetivo-geral}

Analisar e comparar as principais ferramentas de desenvolvimento de shaders dentre as game engines especificadas no escopo deste trabalho com foco na otimização de performance em cada uma, identificando os processos-chave característicos de construção e execução de shaders.

\subsection{Objetivos Específicos}
\label{sec:objetivos-especificos}

	\begin{alineas}
		\item Discriminar as ferramentas de criação de shader de cada game engine bem como suas características individuais.
		\item Determinar os indicadores que serão utilizados para mensurar os parâmetros que serão avaliados nos testes dos shaders.
		\item Desenvolver um “cenário” padrão que possa ser aplicado aos shaders a serem testados.
		\item Realizar testes de performance dos shaders para cada game engine.
		\item Avaliar os resultados obtidos após a conclusão dos testes.
	\end{alineas}

\section{Estrutura do trabalho}
\label{sec:estrutura}

Este estudo está divido em cinco capítulos. O primeiro capítulo contém uma breve explanação do conteúdo introdutório, que detalha o problema de pesquisa, delimita o objetivo geral e enumera os objetivos específicos.

Já no segundo capítulo, há uma exposição da revisão bibliográfica associada ao estudo, explicando conceitos fundamentais para seu entendimento como o desenvolvimento dos shaders, o funcionamento da API de renderização OpenGL, as ferramentas de motores de jogos, os conceitos técnicos de shaders e as principais formas de otimização.

O terceiro capítulo descreve a metodologia utilizada para a realização do trabalho e contém as etapas para o delineamento da sequência lógica do estudo. O quarto capítulo apresenta os objetos desse estudo e os dados pertinentes.

O quinto capítulo abrange a conclusão do trabalho, expondo as considerações finais; sumarizando os resultados do estudo e as ponderações a respeito dos dados e informações demonstradas. Por último, nas referencias bibliográficas, estão especificadas todas as fontes, além de seus devidos autores, aplicadas para a realização deste trabalho.
