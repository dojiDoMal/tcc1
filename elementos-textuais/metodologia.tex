\chapter{Metodologia}
\label{chap:metodologia}

O propósito dessa seção é especificar como o trabalho foi realizado, apresentando o tipo de pesquisa, o meio de coleta de dados, o cenário e a análise dos dados coletados a fim de atingir os objetivos propostos.

\section{Fluxo de Desenvolvimento}
\label{sec:fluxo-de-desewnvolvimento}

A figura \ref{fig:fluxo} mostra o fluxo utilizado para o desenvolvimento deste trabalho.

\begin{figure}[h!]
	\centering
	\Caption{\label{fig:fluxo} Fluxograma do processo}	
	\UNIFORfig{}{
		\fbox{\includegraphics[width=9cm]{figuras/fluxo}}
	}{
		\Fonte{Elaborado pelo autor (2021)}
	}	
\end{figure}

\section{Sobre a pesquisa}
\label{sec:sobre-a-pesquisa}

A pesquisa realizada para a elaboração do presente trabalho é do tipo aplicada, por se tratar de um estudo de análise comparativa e que propõe gerar conhecimento. Esse é um tipo de pesquisa que envolve estudos profundos com a intenção de resolver problemas presentes no em um contexto social semelhante ao dos pesquisadores (GIL, 2017)\nocite{pesquisa}.

Em relação a classificação da pesquisa, pode-se constatar que trata-se de um estudo com características exploratórias por proporcionar maior familiaridade com o problema e descritivas por focar nas características do objeto de estudo (GIL, 2017). Sua finalidade é portanto de criar e ampliar pressuposições, elucidar informações e incertezas sobre o assunto e complementar os entendimentos do expolorador.

Já quanto ao tipo de abordagem, seguindo a definição de Hernandez et. al. (2013)\nocite{hernandez2013}, entende-se que é do tipo quantitativa por ocorrer por meio da coleta de dados para teste de hipóteses baseando-se na medição numérica e na análise estatística para comprovar teorias e estabelecer padrões. Na abordagem quantitativa, os resultados são propagados por meio de quantitativos adquiridos no processo de coleta dos dados.

Conforme Marconi e Lakatos (2019)\nocite{marconi2019}, a investigação bibliográfica é feita pela pesquisa e exposição de bibliografias públicas (livros, revistas, artigos, teses), para mostrar com profundidade os assuntos delimitados pelo pequisador. A pesquisa bibliográfica, dessa forma, ajuda a obter resultados expressivos no estudo desenvolvido. 

\section{Escolha das game engines}
\label{sec:escolha-das-game-engines}

Em relação a Unity, é a engine mais popular entre desenvolvedores de jogos (motor mais popular na plataforma de jogos Steam), especialmente para projetos pequenos e médios. Alguns jogos populares desenvolvidos com ela são: Fall Guys, Among Us, Phasmophobia e Cities (DOUCET; PECORELLA, 2021)\nocite{lars2021}.

Já sobre a Unreal, que cada vez mais reduz as taxas de licenciamento e torna-se mais acessível. Ela é mais propícia para projetos de grande porte (grandes estúdios e jogos AAA). Jogos famosos desenvolvidos com essa engine: ARK, Borderlands, XCOM e PUBG (DOUCET; PECORELLA, 2021).

Por outro lado, a escolha da Godot ocorreu por ser uma engine intuitiva, de código aberto, que apresenta diversas funções que facilitam o desenvolvimento de jogos e por apresentar um crescimento de popularidade e de uso entre desenvolvedores (VARGAS, 2020)\nocite{vargas2020}. 

\section{Sobre os dados}
\label{sec:sobre-os-dados}

Neste trabalho, os dados necessários para a realização do \textit{benchmarking}, como taxa de quadros e tempo de execução, foram obtidos a partir da execução das ferramentas de \textit{profiling}. Pois são dados que são obtidos em tempo real. Esses dados foram analisados para os shaders utilizados no cenário base em cada uma das game engines levando em consideração múltiplos níveis de otimização.

Em relação as características de hardware, foi utilizado um computador com sistema operacional Windows 7 Ultimate, processador Intel(R) Core(TM) i3-2120 CPU @ 3.30GHz, memória RAM de 8 GB, disco rígido de 500 GB, unidade de estado sólido de 120 GB, placa de vídeo PCI Radeon R7 360 e placa mãe com chipset H61.
 