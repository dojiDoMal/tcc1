\chapter{Metodologia}
\label{chap:metodologia}

O propósito dessa seção é especificar como o trabalho foi realizado, apresentando o tipo de pesquisa, o meio de coleta de dados, o cenário e a análise dos dados coletados a fim de atingir os objetivos propostos.

\section{Tipo de pesquisa}
\label{sec:tipo-de-pesquisa}

A pesquisa realizada para a elaboração do presente trabalho é do tipo aplicada, por se tratar de um estudo de análise comparativa e que propõe gerar conhecimento. Esse é um tipo de pesquisa que envolve estudos profundos com a intenção de resolver problemas presentes no em um contexto social semelhante ao dos pesquisadores (GIL, 2017)\nocite{pesquisa}.

Em relação a classificação da pesquisa, pode-se constatar que trata-se de um estudo com características exploratórias por proporcionar maior familiaridade com o problema e descritivas por focar nas características do objeto de estudo (GIL, 2017).

Já quanto ao tipo de abordagem, seguindo a definição de Hernandez et. al. (2013)\nocite{hernandez2013}, entende-se que é do tipo quantitativa por ocorrer por meio da coleta de dados para teste de hipóteses baseando-se na medição numérica e na análise estatística para comprovar teorias e estabelecer padrões. 